%%%%%%%%%%%%%%%%%%%%%%%%%%%%%%%%%%%%%%%%%%%%%%%%%%%%%%%%%%%%%%%%%%%%%%%%%%%%%%%%
%2345678901234567890123456789012345678901234567890123456789012345678901234567890
%        1         2         3         4         5         6         7         8

\documentclass[letterpaper, 10 pt, conference]{ieeeconf}  % Comment this line out
                                                          % if you need a4paper
%\documentclass[a4paper, 10pt, conference]{ieeeconf}      % Use this line for a4
                                                          % paper

\IEEEoverridecommandlockouts                              % This command is only
                                                          % needed if you want to
                                                          % use the \thanks command
\overrideIEEEmargins
% See the \addtolength command later in the file to balance the column lengths
% on the last page of the document

\usepackage[utf8]{inputenc}
\usepackage[T1]{fontenc}
\usepackage{graphicx}
% The following packages can be found on http:\\www.ctan.org
%\usepackage{graphics} % for pdf, bitmapped graphics files
%\usepackage{epsfig} % for postscript graphics files
%\usepackage{mathptmx} % assumes new font selection scheme installed
%\usepackage{mathptmx} % assumes new font selection scheme installed
%\usepackage{amsmath} % assumes amsmath package installed
%\usepackage{amssymb}  % assumes amsmath package installed

\title{\LARGE \bf
\textit{Deep Learning} na Detecção de Pneumonia A partir de Áudios Extraídos de Estetoscópios Digitais
}


\author{Arthur Henrique da Silva}

\begin{document}
\bibliography{refs.bib}


\maketitle
\thispagestyle{empty}
\pagestyle{empty}


%%%%%%%%%%%%%%%%%%%%%%%%%%%%%%%%%%%%%%%%%%%%%%%%%%%%%%%%%%%%%%%%%%%%%%%%%%%%%%%%
\section{Introdução}

A pneumonia é um acometimento no parênquima pulmonar causada por grande variedade de agentes, incluindo bactérias, micoplasma, fungos, parasitas e
vírus, sendo a pneumonia bacteriana a causa mais comum da doença, e o assunto abordado para o reconhecimento desta patologia é a ausculta pulmonar, que é um método semiológico básico no exame dos pulmões, onde é possível reconhecer padrões baseado em regiões e sons pulmonares através de equipamentos médicos, o estetoscópio digital, que além de permitir um exame detalhado, registra os dados em sinais de áudios. E, com a abordagem deep learning é possível analisar com precisão e correlacionar doenças e padrões da mecânica dos pulmões, tornando o diagnóstico assertivo através de um classificador para o método de ausculta pulmonar.


\section{Definição do Problema}

Construção de um classificador de pneumonia de tipo viral ou bacteriana através de áudios extraídos de estetoscópios digitais com abordagem 
\textit{deep learning} para detecção da doença. Estes áudios respiratórios extraídos são organizados pelo número do paciente, idade, sexo, massa corpórea, peso e altura; e coletados a partir da região da traqueia, peito anterior esquerdo, peito anterior direito, peito posterior esquerdo, peito anterior direito, peito lateral esquerdo e peito lateral direito, no ciclo de respiração completo, e considerando sons anormais na ausculta pulmonar e.g: (roncos, sibilos, estridores, grasnidos), caso haja.

Onde os métodos de extração de dados através de Mel-Frequency Cepstral Coefficients (MFCC) e short-time Fourier transform (STFT) e, comparados a partir de espectogramas.

\begin{figure}[htp]
    \centering
    \includegraphics[width=4cm]{spectogram.png}
    \caption{Espectograma de áudio de Indivíduo com pneumonia}
    \label{fig:galaxy}
\end{figure}

\section{Hipótese}

A partir do método semiológico de ausculta pulmonar, classificar tipos de pneumonias através de áudios respiratórios extraídos de estetoscópios digitais.


\section{Objetivo de Pesquisa}

Definir e implementar um modelo de classificação de pneumonia bacteriana ou viral utilizando estetoscópios digitais baseado em áudios e coletar métodos para classificação existentes para detecção da doença a partir de abordagem \textit{deep learning}.

\begin{itemize}
\item Coletar métodos de ausculta pulmonar com abordagem deep learning.

\item Definir o modelo classificatório para análise de padrões e sinais da doença.

\item Análisar os métodos e atualização dos modelos.

\end{itemize}
\ \\

\section{Periódicos (Journals e Magazines)}
\begin{itemize}
\item EURASIP Journal on Image and Video Processing
\item JOURNAL OF RESPIRATORY AND CRITICAL CARE MEDICINE
\item IEEE Transactions on Audio Speech and Language Processing
\item IEEE Transactions on Biomedical Engineering
\ \\

\end{itemize}
\section{Conferências e Simpósios}

\begin{itemize}
\item Proceedings of the International Symposium on Auditory and Audiological Research
\item Signal Processing Conference (EUSIPCO), European
\end{itemize}


\begin{thebibliography}{99}

\bibitem{c1}
Leng, S., Tan, R., Chai, K., Wang, C., Ghista, D. Zhong, L. The electronic stethoscope.  Biomedical Engineering Online.  (2015)

\bibitem{c2}
Toews, G. Southwestern internal medicine conference: nosocomial pneumonia.  The American Journal Of The Medical Sciences.  (1986)

\bibitem{c3}
Nolte, F. Molecular Diagnostics for Detection of Bacterial and Viral Pathogens in Community-Acquired Pneumonia. Clinical Infectious Diseases.  (2008)

\bibitem{c4}
Figueiredo, L. Pneumonias virais: aspectos epidemiológicos, clinícos, fisiopatológicos e tratamento. Jornal Brasileiro De Pneumologia. (2009)

\bibitem{c5}
Bohadana, A., Izbicki, G. and Kraman, S. Fundamentals of Lung Auscultation. The New England Journal Of Medicine. (2014)

\bibitem{c6}
Rocha, B Respiratory Sound Database for the Development of Automated Classification.  (2018)

\bibitem{c7}
Kilic, Ozkan and Kılıç, Özkan and Kurt, Bahar and Saryal, Sevgin. 
Classification of lung sounds using convolutional neural networks (2017)

\end{thebibliography}




\end{document}