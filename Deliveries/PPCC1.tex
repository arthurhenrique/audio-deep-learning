\documentclass[12pt]{article}

\usepackage{amsmath}    % need for subequations
\usepackage{graphicx}   % need for figures
\usepackage{verbatim}   % useful for program listings
\usepackage{color}      % use if color is used in text

\usepackage{hyperref}   % use for hypertext links, including those to external documents and URLs

% don't need the following. simply use defaults
\setlength{\baselineskip}{16.0pt}    % 16 pt usual spacing between lines

\setlength{\parskip}{3pt plus 2pt}
\setlength{\parindent}{20pt}
\setlength{\oddsidemargin}{0.5cm}
\setlength{\evensidemargin}{0.5cm}
\setlength{\marginparsep}{0.75cm}
\setlength{\marginparwidth}{2.5cm}
\setlength{\marginparpush}{1.0cm}
\setlength{\textwidth}{150mm}

\begin{comment}
\pagestyle{empty} % use if page numbers not wanted
\end{comment}

% above is the preamble

\begin{document}

\begin{center}
{\large Classificar Pneumonia Bacteriana e Viral A partir de \'Audios Respirat\'orios Extra\'idos de Estetosc\'opios Digitais} \\  % \\ = new line
\ \\

Arthur Henrique da Silva \\
Outubro, 2019

\end{center}

\section{Problema Tratado}

Construir um modelo de classifica\c{c}\~ao de pneumonia bacteriana e viral baseado em \'audios extra\'idos de estesc\'opios digitais atrav\'es de m\'etodos de
Rede de Aprendizagens Profundas (RNN) para an\'alise dos sinais de \'audio coletados.

\section{Hip\'otese}
Classificar pneumonia bacteriana e viral a partir de \'audios respirat\'orios extra\'idos de estesc\'opios digitais.

\section{Objetivo de pesquisa}
Definir e implementar um modelo de classifica\c{c}\~ao de pneumonia bacteriana e viral utilizando estesc\'opios digitais e compreender m\'etodos para classifica\c{c}\~ao existentes.

\section{Peri\'odicos (journals e magazines)}
https://signalprocessingsociety.org/publications-resources/ieee-journal-selected-topics-signal-processing

\section{Confer\^encias e Simp\'osios}
https://www.myhuiban.com/conference/2854?lang=es


\end{document}
