
\documentclass[12pt]{article}

\usepackage{multicol}
\setlength{\columnsep}{1cm}

 
\usepackage{amsmath}    % need for subequations
\usepackage{graphicx}   % need for figures
\usepackage{verbatim}   % useful for program listings
\usepackage{color}      % use if color is used in text

\usepackage{hyperref}   % use for hypertext links, including those to external documents and URLs

% don't need the following. simply use defaults
\setlength{\baselineskip}{16.0pt}    % 16 pt usual spacing between lines

\setlength{\parskip}{3pt plus 2pt}
\setlength{\parindent}{20pt}
\setlength{\oddsidemargin}{0.5cm}
\setlength{\evensidemargin}{0.5cm}
\setlength{\marginparsep}{0.75cm}
\setlength{\marginparwidth}{2.5cm}
\setlength{\marginparpush}{1.0cm}
\setlength{\textwidth}{150mm}


% above is the preamble

\begin{document}
\begin{center}
{\large \textit{Deep Learning} na Detecção de Pneumonia A partir de Áudios Extraídos de Estetoscópios Digitais} \\  % \\ = new line
\ \\
Arthur Henrique da Silva \\
\end{center}

\begin{multicols}{2}

\section{Problema Tratado}
Construção de um classificador de pneumonia de tipo viral ou bacteriana através de áudios extraídos de estetoscópios digitais com abordagem 
\textit{deep learning} para detecção da doença. Estes áudios respiratórios extraídos são organizados pelo número do paciente, idade, sexo, massa corpórea, peso e altura; e coletados a partir da região da traqueia, peito anterior esquerdo, peito anterior direito, peito posterior esquerdo, peito anterior direito, peito lateral esquerdo e peito lateral direito, no ciclo de respiração completo, e considerando crepitações e chiados, caso haja.

\section{Hipótese}
Detectar pneumonia bacteriana ou viral a partir de áudios respiratórios extraídos de estetoscópios digitais.
\ \\
\section{Objetivo de Pesquisa}

Definir e implementar um modelo de classificação de pneumonia bacteriana ou viral utilizando estetoscópios digitais baseado em áudios e coletar métodos para classificação existentes para detecção da doença a partir de abordagem \textit{deep learning}.

\section{Periódicos (Journals e Magazines)}
\begin{itemize}
    \item \link{https://signalprocessingsociety.org/publications-resources
    /ieee-journal-selected-topics-signal-processing}
    \item \link{https://ieeexplore.ieee.org/xpl/
    RecentIssue.jsp?punumber=79}

\end{itemize}

\section{Conferências e Simpósios}
\begin{itemize}
    \item \link{https://www.myhuiban.com/conference
    /2854?lang=es}
    \item \link{https://ieeexplore.ieee.org/xpl/
    conhome/6937054/proceeding}
\end{itemize}

\end{multicols}

\end{document}
